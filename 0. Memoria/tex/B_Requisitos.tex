\apendice{Especificación de Requisitos}

\section{Introducción}

En este apartado se exponen los principales requisitos del proyecto.

\section{Objetivos generales}

Estos son los objetivos que se marcaron en el proyecto:

\begin{itemize}
    \item Replicar el trabajo \textit{Innate and Adaptive Immune Assessment at Admission to Predict Clinical Outcome in COVID-19 Patients}~\cite{sansegundo:2021}.
    \item Buscar trabajos relacionados y clasificar las variables, técnicas y herramientas que se utilizaron.
	\item Estudiar los datos proporcionados por el Instituto de Investigación Sanitaria de Valdecilla (IDIVAL) y considerar como procesarlos para poder obtener un un modelo de predicción que mejore los rendimientos del trabajo \textit{Innate and Adaptive Immune Assessment at Admission to Predict Clinical Outcome in COVID-19 Patients}~\cite{sansegundo:2021}.
    \item Utilizar librerías de aprendizaje automático que ayuden a encontrar el mejor método de clasificación de entre diferentes métodos a partir de las variables/marcadores para el Covid-19 y disponibles en el conjunto de datos.
    \item Minimizar las características a utilizar para su utilización en el modelo a desarrollar.
    \item Investigar qué hiperparámetros pueden mejorar el método de clasificación elegido para el modelo final.
\end{itemize}

\section{Catálogo de requisitos}

Debido al carácter fundamentalmente experimental de nuestro proyecto, distinguiremos en requisitos funcionales (qué realizan los modelos desarrollados) y no funcionales (cómo lo consigue).

\subsection{Requisitos funcionales}
\begin{itemize}

    \item \textbf{RF-1} Los datos deben poder ser divididos en conjuntos de entrenamiento y validación.
        \begin{itemize}
            \item \textbf{RF-1.1} Se indicará qué porcentaje de datos del total se utilizará en el conjunto de validación.
            \item \textbf{RF-1.2} Aleatorización de la inclusión de instancias en el conjunto de entrenamiento o validación, permitiendo siempre una división paritaria en ambos conjuntos.        
        \end{itemize}
    
    \item \textbf{RF-2} Se debe poder aplicar técnicas y algoritmos de preparación de los datos.
        \begin{itemize}
            \item \textbf{RF-2.1} Se debe poder eliminar los valores nulos de la característica a entrenar.
            \item \textbf{RF-2.2} Se debe poder imputar los valores perdidos o nulos en el resto del conjunto de datos
        \end{itemize}
    
    \item \textbf{RF-3} Se debe poder entrenar un modelo.
        \begin{itemize}
            \item \textbf{RF-3.1} Ofrecer la posibilidad de escoger entre diferentes algoritmos de clasificación.
            \item \textbf{RF-3.2} Debe guardarse el mejor modelo generado durante el entrenamiento.
            \item \textbf{RF-3.3} Se debe poder mostrar gráficamente los resultados del entrenamiento.
        \end{itemize}

    \item \textbf{RF-4} Se debe poder realizar búsquedas de mejor clasificador de forma automática.
        \begin{itemize}
            \item \textbf{RF-4.1} Se debe poder cambiar el dataset a utilizar.
            \item \textbf{RF-4.2} Se debe poder cambiar los parámetros de búsqueda.
            \item \textbf{RF-4.3} Se debe obtener una solución gráfica en la que observar los resultados.
        \end{itemize}

    \item \textbf{RF-5} Se debe poder optimizar los hiperparámetros de un modelo.
        \begin{itemize}
            \item \textbf{RF-5.1} Se debe poder cambiar el modelo a optimizar.
            \item \textbf{RF-5.2} Se debe obtener un gráfico con los resultados.
        \end{itemize}

    \item \textbf{RF-6} Se debe poder probar un modelo guardado.
        \begin{itemize}
            \item \textbf{RF-6.1} Se debe poder encontrar el modelo a cargar.
        \end{itemize}
\end{itemize}

\subsection{Requisitos no funcionales}

\begin{itemize}
    \item \textbf{RNF-1} Portabilidad: Se puede ejecutar entre diferentes sistemas operativos y entornos de desarrollo, lo que evita problemas de configuración y compatibilidad.
    \item \textbf{RNF-2} Reproducibilidad: El contenedor contiene todas las dependencias y configuraciones específicas, lo que facilita la reproducción de resultados y la colaboración.
    \item \textbf{RNF-2} Aislamiento: Los paquetes o bibliotecas instalados en el contenedor no afectarán a tu sistema host, y viceversa.
    \item \textbf{RNF-2} Escalabilidad: Permite manejar grandes conjuntos de datos y ejecutar cálculos intensivos de manera eficiente utilizando recursos distribuidos.
    \item \textbf{RNF-2} Administración simplificada: Gestión de múltiples Jupyter Notebooks y entornos. Se pueen crear y administrar diferentes contenedores para proyectos específicos o diferentes configuraciones de entorno.
    \item \textbf{RNF-2} Flexibilidad: Permite personalizar y configurar el entorno para instalar las bibliotecas y herramientas necesarias, configurar extensiones y temas, y ajustar los recursos del contenedor para optimizar el rendimiento.
\end{itemize}
\newpage

\section{Especificación de requisitos}

\subsection{Diagrama de caso de uso}

\imagen{diaguso}{Diagrama de caso de uso}

\subsection{Especificaciones de los casos de uso}

\tablaSinCabeceraConBandas{Caso de Uso \textit{División de datos en conjuntos de entrenamiento y validación}}{p{3cm} p{9cm}}{2}{casouso1}
{Caso de uso & División de datos en entrenamiento y validación \\ 
Requisitos &  RF-1\newline
            RF-1.1\newline
            RF-1.2 \\ 
Descripción & Los datos deben poder ser divididos en conjuntos de entrenamiento y validación. \\ 
Precondiciones & Se debe disponer de un conjunto de datos. \\ 
Acciones & 1. Se indicará qué porcentaje de datos del total se utilizará en el conjunto de validación. \newline
            2. Aleatorización de la inclusión de instancias en el conjunto de entrenamiento o validación, permitiendo siempre una división paritaria en ambos conjuntos. \\ 
Postcondiciones & Datos divididos en dos conjuntos.\\ 
Excepciones & El conjunto no lleva datos o está corrupto. \\ 
Importancia & Media\\
}

\tablaSinCabeceraConBandas{Caso de Uso \textit{Aplicar técnicas y algoritmos de preparación de los datos}}{p{3cm} p{9cm}}{2}{casouso2}
{Caso de uso & Preparación de datos \\ 
Requisitos &  RF-2\newline
            RF-2.1\newline
            RF-2.2 \\ 
Descripción & Los datos deben poder ser divididos en conjuntos de entrenamiento y validación. \\ 
Precondiciones & Se debe disponer de un conjunto de datos. \\ 
Acciones & 1.  Se debe poder eliminar los valores nulos de la característica a entrenar. \newline 
            2. Se debe poder imputar los valores perdidos o nulos en el resto del conjunto de datos. \\ 
Postcondiciones & Ddatos preparados.\\ 
Excepciones & El conjunto no lleva datos o está corrupto. \\ 
Importancia & Media\\
}

\tablaSinCabeceraConBandas{Caso de Uso \textit{Entrenar un modelo}}{p{3cm} p{9cm}}{2}{casouso3}
{Caso de uso & Entrenar un modelo \\ 
Requisitos &  RF-3\newline
            RF-3.1\newline
            RF-3.2\newline
            RF-3.3 \\ 
Descripción & Se debe poder entrenar un modelo. \\ 
Precondiciones & Se debe disponer de un conjunto de datos. \\ 
Acciones & 1. Ofrecer la posibilidad de escoger entre diferentes algoritmos de clasificación. \newline
           2. Debe guardarse el mejor modelo generado durante el entrenamiento. \newline
           3. Se debe poder mostrar gráficamente los resultados del entrenamiento. \\ 
Postcondiciones & Modelo entrenado.\\ 
Excepciones & El conjunto no lleva datos o está corrupto. \\ 
Importancia & Alta\\
}

\tablaSinCabeceraConBandas{Caso de Uso \textit{Búsqueda de mejor clasificador de forma automática.}}{p{3cm} p{9cm}}{2}{casouso4}
{Caso de uso & Búsqueda automática de mejor clasificador \\ 
Requisitos &  RF-4\newline
            RF-4.1\newline
            RF-4.2\newline
            RF-4.3 \\ 
Descripción & Se debe poder realizar búsquedas de mejor clasificador de forma automática. \\ 
Precondiciones & Se debe instalar la librería de automl. \\ 
Acciones & 1. Se debe poder cambiar el dataset a utilizar. \newline
           2. Se debe poder cambiar los parámetros de búsqueda. \newline
           3. Se debe obtener una solución gráfica en la que observar los resultados. \\ 
Postcondiciones & Mejor clasificador encontrado.\\ 
Excepciones & Error al realizar la tarea de búsqueda. \\ 
Importancia & Media\\
}

\tablaSinCabeceraConBandas{Caso de Uso \textit{Optimización de hiperparámetros.}}{p{3cm} p{9cm}}{2}{casouso5}
{Caso de uso & Optimización de hiperparámetros \\ 
Requisitos &  RF-5\newline
            RF-5.1\newline
            RF-5.2 \\ 
Descripción & Se debe poder optimizar los hiperparámetros de un modelo. \\ 
Precondiciones & Se debe instalar la librería de optimización de hiperparámetros. \\ 
Acciones & 1. Se debe poder cambiar el modelo a optimizar. \newline
           2. Se debe obtener un gráfico con los resultados. \\ 
Postcondiciones & Mejores hiperparámetros encontrados.\\ 
Excepciones & Error al realizar la tarea de optimización. \\ 
Importancia & Media\\
}

\tablaSinCabeceraConBandas{Caso de Uso \textit{Prueba de modelos}}{p{3cm} p{9cm}}{2}{casouso6}
{Caso de uso & Prueba de modelos \\ 
Requisitos &  RF-6\newline
            RF-6.1 \\ 
Descripción & Se debe poder probar un modelo guardado. \\ 
Precondiciones & Modelos desarrollados alojados correctamente \\ 
Acciones & 1. Se debe poder encontrar el modelo a cargar. \\ 
Postcondiciones & Prueba correcta de modelo\\ 
Excepciones & No se encuentra el modelo en la ubicación \\ 
Importancia & Media\\
}