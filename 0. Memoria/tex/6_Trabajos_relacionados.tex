\capitulo{6}{Trabajos relacionados} \label{trabajosrel}

\section{Búsqueda de trabajos relacionados}
Investigación en los principales buscadores de artículos científicos acerca de documentos publicados relacionados con palabras como COVID-19, clasificación, aprendizaje automático, triaje y score. De entre todos, se discriminaron por los artículos que se basaran en \textit{datasets} que contuviesen marcadores recogidos a los pacientes para el diagnóstico del COVID-19.

En la tabla \ref{tabla:trabajosrelacionados}, se enumeran el listado de los trabajos relacionados con la fecha de publicación y una breve descripción.

\tablaConColores{Trabajos relacionados con este proyecto}{p{4cm} p{1.7cm} p{6cm}}{3}{trabajosrelacionados}
{Título & Fecha & Descripción\\}{ 
Innate and Adaptive Immune Assessment at Admission to Predict Clinical Outcome in COVID-19 Patients \cite{sansegundo:2021} & 29/07/2021 & Predicción de la severidad de la enfermedad en la admisión del paciente categorizada por los requisitos de oxigenoterapia\\
Machine-learning-based COVID-19 mortality prediction model and identification of patients at low and high risk of dying \cite{banoei:2021} & 08/09/2021 & Modelo de predicción de mortalidad para pacientes COVID-19 hospitalizados, así como una clasificación de pacientes para verificar los grupos de bajo y alto riesgo\\
Deep forest model for diagnosing COVID-19 from routine blood tests \cite{aljame:2021} & 17/08/2021 & Uso de técnicas de machine learning basadas en datos clínicos y/o de laboratorio para la detección temprana de COVID-19\\
Development of a severity of disease score and classification model by machine learning for hospitalized COVID-19 patients \cite{marcos:2021} & 21/04/2021 & Modelo de machine learning para identificar de manera temprana a los pacientes que morirán o requerirán ventilación mecánica durante la hospitalización a partir de las características clínicas y de laboratorio obtenidas en el momento del ingreso\\
Machine learning based predictors for COVID-19 disease severity \cite{patel:2021} & 25/02/2021 & Clasificación mediante random forest de la severidad en los pacientes con COVID-19 para calcular si necesitarán ventilación mecánica o/y ingreso en UCI\\
Prediction model and risk scores of ICU admission and mortality in COVID-19 \cite{zhao:2020} & 30/07/2020 & Desarrolla scores basados en las características clínicas en el momento de la admisión para predecir el ingreso y la mortalidad en UCI en pacientes con COVID-19\\
Scoring systems for predicting mortality for severe patients with COVID-19 \cite{shang:2020} & 15/07/2020 & Definición de sistemas de score para predecir el ingreso en UCI y/o la mortalidad en pacientes de COVID-19\\
A Multimodality Machine Learning Approach to Differentiate Severe and Nonsevere COVID-19: Model Development and Validation \cite{chen:2021} & 07/04/2021 & Uso de machine learning para diferenciar con precisión los tipos clínicos de COVID-19 graves y no graves en función de múltiples características médicas y proporcionar predicciones fiables del tipo clínico de la enfermedad\\
Multivariable mortality risk prediction using machine learning for COVID-19 patients at admission (AICOVID) \cite{kar:2021} & 17/06/2021 & Desarrollar y validar puntuaciones de riesgo de mortalidad individualizadas basadas en los datos clínicos y de laboratorio anonimizados en el momento del ingreso y determinar la probabilidad de Muertes a los 7 y 28 días\\
A systematic review of prediction models to diagnose COVID-19 in adults admitted to healthcare centers \cite{locquet:2021} & 18/06/2021 & Identificar, comparar y evaluar el desempeño de los modelos de predicción para el diagnóstico de COVID-19 en pacientes adultos en un entorno de atención médica\\
Machine learning is the key to diagnose COVID-19: a proof-of-concept study \cite{gangloff:2021} & 30/03/2021 & Desarrollar y evaluar modelos de aprendizaje automático utilizando datos clínicos y de laboratorio de rutina para mejorar el rendimiento de RT-PCR y CT de tórax para el diagnóstico de COVID-19 entre pacientes hospitalizados después de una urgencia\\
Comparing machine learning algorithms for predicting ICU admission and mortality in COVID-19 \cite{subudhi:2021} & 21/05/2021 & Compara el rendimiento de 18 algoritmos de aprendizaje automático para predecir la admisión y la mortalidad en la UCI entre pacientes con COVID-19\\
Rapid identification of SARS-CoV-2-infected patients at the emergency department using routine testing \cite{kurstjens:2021} & 29/06/2020 & Desarrollar un algoritmo para evaluar rápidamente el riesgo de una persona de infección por SARS-CoV-2 en el servicio de urgencias\\
Machine Learning for Mortality Analysis in Patients with COVID-19 \cite{sanchez:2021} & 12/11/2020 & Modelos de ML para encontrar reglas de decisión interpretables para estimar el riesgo de mortalidad de los pacientes que se pueden obtener del árbol de decisiones y que puede ser crucial en la priorización de la atención y los recursos médicos\\
}

\section{Clasificación de variables, técnicas y herramientas}

Una vez escogidos los trabajos sobre los que se basa la investigación para este proyecto, aparece la necesidad de clasificar de cada trabajo qué variables, técnicas y herramientas utilizaron los diversos autores en su investigación. De esta manera, se da la posibilidad de ver qué variables son las más usadas en las investigaciones, así como qué técnicas más se repiten y cuáles son las herramientas que se utilizan.

En la siguiente tabla \ref{tabla:clasificavariables} se expone qué variables con qué técnicas y herramientas se utilizaron en los trabajos relacionados.

\tablaConColores{Clasificación de variables, técnicas y herramientas}{p{5cm} p{3.4cm} p{3.4cm}}{3}{clasificavariables}
{Variables & Técnicas & Herramientas\\}{ 
Age, Ferritin, D-dimer, C4, \% de $CD8^{+}CD27^{-}CD28^{-}$, Suma absoluta de Linfocitos, \% de Monocitos no clásicos & Regresión Logística & GraphPad (Prism)\\
AST, WBC, Lymphocytes, Neutrophils, GGT, AGE, Basophils, Eosinophils, ALT, Platelets, gender, CRP, ALP, LDH, Monocytes & Extra trees, XGBoost, LightGBM, SHapley Additive exPlanations (SHAP) & Python (sklearn) \\
Old age, coronary heart disease (CHD), percentage of lymphocytes (LYM\%), procalcitonin (PCT), D-dimer (DD) & LASSO regression, Roc curve & Python\\
demographic variables (including age and sex), individual comorbidities and Charlson Comorbidity Index, chronic medical treatment, clinical characteristics, physical examination parameters, biochemical parameters & Random forest, Xgboost, Logistic regression, AUC & Python (sklearn, xgboost, eli5) \\
Leukocytes, Neutrophils, Lymphocytes, Eosinophil, Hemoglobin, Hematocrit, Platelets, ESR, BUN, Creatinine, Na, K, Ferritin, CRP, PCT, Lactate, Troponin, CK, BNP, LDH, Fibrinogen, ALT, AST, Albumin, D-dimer, Bilirubin, Prothrombin, APTT, pH, PaCo2, FiO2\_lab, Bicarbonate, CAD, Diabetes, Age > 65, AMS, Dementia, Nursing home, Q2 saturation < 88, yno2, Consolidation, Hypertension, Atrial fibrillation, Alcohol, Chest pain, Peripheral vascular disease, Stroke, Headache, Dyspnea, CRP, Lactate, Smoking & SIMPLS, AUC (Area under ROC curve), Decission Tree &	Python \\
age, sex, travel, contact history, and co-morbidities, fever, dyspnea, respiratory rate, and blood oxygen saturation (SpO2), RT-PCR, InterLeukin-6, D-Dimer, complete blood count, lipase, and C-reactive protein (CRP) & Random Forest, multilayer perceptron, support vector machines, gradient boosting, extra tree classifier, adaboost, Regresión, Validación cruzada quíntuple, AUC & Python \\
lactate dehydrogenase, procalcitonin, pulse oxygen saturation, smoking history, lymphocyte count, heart failure, chronic obstructive pulmonary disease, heart rate, age & Regresión Logística, AUC & TRIPOD \\
age, hypertension, cardiovascular disease, gender, diabetes, dimerized plasmin fragment D, high sensitivity troponin I, absolute neutrophil count, interleukin 6, lactate dehydrogenase & Random Forest, Predictive mean matching (PMM), Regresión Logística, AUC & R, Python \\
Age, Male Gender, Respiratory Distress, Diabetes Mellitus, Chronic Kidney Disease, Coronary Artery Disease, respiratory rate > 24/min, oxygen saturation below 90\%, Lymphocyte\% in DLC, INR, LDH, Ferritin & eXtreme Gradient Boosting (XGB), Random Forest, Regresión Logística, ‘Time to event’ using Cox Proportional Hazard Model, AUC ROC, Kaplan Meier (KM) Plots & Python \\
 & Prediction model study Risk of Bias Assessment Tool (PROBAST), XGBoost, Regresión Logísitica, AUROC & Python \\
Cough, Hyperthermia, Myalgias, Asthenia, Diarrhea, Confusion, Furosemid (usual treatment), Base excess, Lactates, Red blood cell count, Mean platelet volume, Leukocytes, Neutrophils, Platelet count, Eosinophils percentage, Basophils percentage, Lymphocytes, Monocytes, Ionogram, Potassium, Phosphor, Alanine aminotransferase, International normalized ratio, D-Dimer & binary logistic regression, random forest, artificial neural network, AUC, K-fold cross-validation & R-Studio (Dplyr, Purrr, missForest, Caret, corrplot, randomForest, neuralnet, pROC) \\
C-reactive protein, neutrophil percentages, lactate dehydrogenase, first respiratory, lower xygen saturation, lymphocyte percentages, estimated glomerular filtration rate (eGFR) < 60, high neutrophil percentage, high serum potassium, low lymphocyte percentages, high procalcitonin, D-dimer & AdaBoostClassifier, BaggingClassifier, GradientBoostingClassifier, RandomForestClassifier, XGBClassifier, ExtraTreesClassifier, LogisticRegression, DecisionTreeClassifier, LinearDiscriminantAnalysis, QuadraticDiscriminantAnalysis, MLPClassifier, PassiveAggressiveClassifier, Perceptron, LinearSVC, KNeighborsClassifier, GaussianNB & Python \\
Age, Gender, C-reactive protein, lactate dehydrogenase, ferritin, absolute neutrophil, lymphocyte counts & AUROC & Python, Excel 2010 \\
Patient ID, Age and gender, COVID diagnostic (confirmed/pending confirmation), ER date in, ER specialty, ER diagnostic, and destination after ER, First and last constant measurements in the ER (heart rate, temperature, minimum and maximum arterial pressure, O2 saturation in blood), Admission date to the hospital, ICU date in, ICU date out, and number of days in the ICU (if applicable), Discharge date and destination (home/deceased/transferred to other hospital/voluntary discharge/transferred to a socio-sanitary center) & Logistic regression, Survival Analisys, Decision tree, Random Forest, Bayessian networks, Biclustering, AUC & R (Caret, e1701, ggplot2), Python (sklearn, pandas, numpy  matplotlib) \\
}