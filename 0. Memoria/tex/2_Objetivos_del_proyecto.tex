\capitulo{2}{Objetivos del proyecto} \label{capi2}

El objeto de este proyecto es desarrollar modelos de aprendizaje automático que posibiliten a los médicos realizar un triaje en la admisión de pacientes con sintomatología COVID-19 a partir de variables/marcadores que se puedan obtener de forma rápida sin tener que esperar a una prueba PCR positiva y que, además, pueda predecir la gravedad de la enfermedad para permitir conocer que recursos hospitalarios (ventilación mecánica, ingreso en UCI, etc...) va a necesitar cada paciente.

\section{Objetivos principales}

\begin{itemize}
	\item Replicar el trabajo \textit{Innate and Adaptive Immune Assessment at Admission to Predict Clinical Outcome in COVID-19 Patients}~\cite{sansegundo:2021}.
    \item Buscar trabajos relacionados y clasificar las variables, técnicas y herramientas que se utilizaron.
	\item Estudiar los datos proporcionados por el Instituto de Investigación Sanitaria de Valdecilla (IDIVAL) y considerar como procesarlos para poder obtener un un modelo de predicción que mejore los rendimientos del trabajo \textit{Innate and Adaptive Immune Assessment at Admission to Predict Clinical Outcome in COVID-19 Patients}~\cite{sansegundo:2021}.
    \item Utilizar librerías de aprendizaje automático que ayuden a encontrar el mejor método de clasificación de entre diferentes métodos a partir de las variables/marcadores para el Covid-19 y disponibles en el conjunto de datos.
    \item Minimizar las características a utilizar para su utilización en el modelo a desarrollar.
    \item Investigar qué hiperparámetros pueden mejorar el método de clasificación elegido para el modelo final.
\end{itemize}

\section{Objetivos personales}

\begin{itemize}
    \item Poner en práctica los conocimientos obtenidos de minería de datos.
    \item Desarrollar el proyecto  de manera colaborativa en Github.
    \item Documentar la memoria del proyecto con \LaTeX.
\end{itemize}