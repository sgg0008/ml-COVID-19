\apendice{Especificación de diseño}

\section{Introducción}

En este apartado recogemos el diseño experimental desarrollado en el proyecto.

\section{Diseño de datos}

La cohorte de datos han sido proporcionados por el Instituto de Investigación Sanitaria de Valdecilla (IDIVAL) sobre una población de 305 pacientes, cuyas muestras fueron recogidas entre Abril de 2020 y Marzo de 2021 y que tuvieron un resultado positivo en COVID-19. El valor de clasificación es el \textbf{score} de evolución clínica que tiene un valor de 1 a 5, siendo 1=asintomático; 2=tratamiento con gafas nasales; 3=tratamiento con ventimask; 4=Ingreso en UCI; 5=exitus.

El dataset se compone de un archivo csv con las siguientes características:
\begin{itemize}
	\item 420 filas o determinaciones.
	\item 97 columnas o marcadores.
    \item Sólo hay 158 filas con el valor de clasificación score cumplimentado.
\end{itemize}

Los valores correspondientes a la intensidad de luz presentan una amplia variedad, moviéndose generalmente entre los intervalos de $-10^5$ a $10^5$. 

\imagen{ejemdatos}{Ejemplo de conjunto de datos}

\section{Diseño procedimental}

En la realización de un nuevo experimento, el científico de datos posee total libertad para elegir las herramientas y las técnicas que va a aplicar. En cualquier caso, suele coincidir de forma general el flujo de trabajo.

El proceso de creación de un experimento lleva habitualmente las siguentes fases: carga de datos, preparación de estos, configuración del modelo y entrenamiento. Puede verse en la imagen \nameref{fig:diagflujo1}.

\imagen{diagflujo1}{Diagrama de flujo de realización de un experimento}

\section{Diseño arquitectónico}