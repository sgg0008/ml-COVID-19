\documentclass[a4paper,11pt,oneside]{memoir}

% Castellano
\usepackage[spanish,es-tabla,es-ucroman]{babel}
\selectlanguage{spanish}
\usepackage{placeins}
\usepackage[utf8]{inputenc}
\usepackage{lmodern}
\usepackage[T1]{fontenc}

\RequirePackage{booktabs}
\RequirePackage[table]{xcolor}
\RequirePackage{xtab}
\RequirePackage{multirow}

% Links
\usepackage[colorlinks]{hyperref}
\hypersetup{
	allcolors = {red}
}

% Ecuaciones
\usepackage{amsmath}

% Rutas de fichero / paquete
\newcommand{\ruta}[1]{{\sffamily #1}}

% Párrafos
\nonzeroparskip


% Imagenes
\usepackage{graphicx}
\newcommand{\imagen}[2]{
	\begin{figure}[!ht]
		\centering
		\includegraphics[width=0.9\textwidth]{#1}
		\caption{#2}\label{fig:#1}
	\end{figure}
	\FloatBarrier
}


\newcommand{\imagenflotante}[2]{
	\begin{figure}%[!ht]
		\centering
		\includegraphics[width=0.9\textwidth]{#1}
		\caption{#2}\label{fig:#1}
	\end{figure}
}



% El comando \figura nos permite insertar figuras comodamente, y utilizando
% siempre el mismo formato. Los parametros son:
% 1 -> Porcentaje del ancho de página que ocupará la figura (de 0 a 1)
% 2 --> Fichero de la imagen
% 3 --> Texto a pie de imagen
% 4 --> Etiqueta (label) para referencias
% 5 --> Opciones que queramos pasarle al \includegraphics
% 6 --> Opciones de posicionamiento a pasarle a \begin{figure}
\newcommand{\figuraConPosicion}[6]{%
  \setlength{\anchoFloat}{#1\textwidth}%
  \addtolength{\anchoFloat}{-4\fboxsep}%
  \setlength{\anchoFigura}{\anchoFloat}%
  \begin{figure}[#6]
    \begin{center}%
      \Ovalbox{%
        \begin{minipage}{\anchoFloat}%
          \begin{center}%
            \includegraphics[width=\anchoFigura,#5]{#2}%
            \caption{#3}%
            \label{#4}%
          \end{center}%
        \end{minipage}
      }%
    \end{center}%
  \end{figure}%
}

%
% Comando para incluir imágenes en formato apaisado (sin marco).
\newcommand{\figuraApaisadaSinMarco}[5]{%
  \begin{figure}%
    \begin{center}%
    \includegraphics[angle=90,height=#1\textheight,#5]{#2}%
    \caption{#3}%
    \label{#4}%
    \end{center}%
  \end{figure}%
}
% Para las tablas
\newcommand{\otoprule}{\midrule [\heavyrulewidth]}
%
% Nuevo comando para tablas pequeñas (menos de una página).
\newcommand{\tablaSmall}[5]{%
 \begin{table}
  \begin{center}
   \rowcolors {2}{gray!35}{}
   \begin{tabular}{#2}
    \toprule
    #4
    \otoprule
    #5
    \bottomrule
   \end{tabular}
   \caption{#1}
   \label{tabla:#3}
  \end{center}
 \end{table}
}

%
% Nuevo comando para tablas pequeñas (menos de una página).
\newcommand{\tablaSmallSinColores}[5]{%
 \begin{table}[H]
  \begin{center}
   \begin{tabular}{#2}
    \toprule
    #4
    \otoprule
    #5
    \bottomrule
   \end{tabular}
   \caption{#1}
   \label{tabla:#3}
  \end{center}
 \end{table}
}

\newcommand{\tablaApaisadaSmall}[5]{%
\begin{landscape}
  \begin{table}
   \begin{center}
    \rowcolors {2}{gray!35}{}
    \begin{tabular}{#2}
     \toprule
     #4
     \otoprule
     #5
     \bottomrule
    \end{tabular}
    \caption{#1}
    \label{tabla:#3}
   \end{center}
  \end{table}
\end{landscape}
}

%
% Nuevo comando para tablas grandes con cabecera y filas alternas coloreadas en gris.
\newcommand{\tablaConColores}[6]{%
  \begin{center}
    \tablefirsthead{
      \toprule
      #5
      \otoprule
    }
    \tablehead{
      \multicolumn{3}{l}{\small continúa desde la página anterior}\\
      \toprule
      #5
      \otoprule
    }
    \tabletail{
      \hline
      \multicolumn{3}{r}{\small continúa en la página siguiente}\\
    }
    \tablelasttail{
      \hline
    }
    \rowcolors{2}{gray!35}{}
    \bottomcaption{#1}
    \begin{xtabular}{#2}
      #6
      \bottomrule
    \end{xtabular}
    \label{tabla:#4}
  \end{center}
}

%
% Nuevo comando para tablas grandes con cabecera.
\newcommand{\tablaSinColores}[6]{%
  \begin{center}
    \tablefirsthead{
      \toprule
      #5
      \otoprule
    }
    \tablehead{
      \multicolumn{#3}{l}{\small continúa desde la página anterior}\\
      \toprule
      #5
      \otoprule
    }
    \tabletail{
      \hline
      \multicolumn{#3}{r}{\small continúa en la página siguiente}\\
    }
    \tablelasttail{
      \hline
    }
    \bottomcaption{#1}
    \begin{xtabular}{#2}
      #6
      \bottomrule
    \end{xtabular}
    \label{tabla:#4}
  \end{center}
}

%
% Nuevo comando para tablas grandes sin cabecera.
\newcommand{\tablaSinCabecera}[5]{%
  \begin{center}
    \tablefirsthead{
      \toprule
    }
    \tablehead{
      \multicolumn{#3}{l}{\small\sl continúa desde la página anterior}\\
      \hline
    }
    \tabletail{
      \hline
      \multicolumn{#3}{r}{\small\sl continúa en la página siguiente}\\
    }
    \tablelasttail{
      \hline
    }
    \bottomcaption{#1}
  \begin{xtabular}{#2}
    #5
   \bottomrule
  \end{xtabular}
  \label{tabla:#4}
  \end{center}
}



\definecolor{cgoLight}{HTML}{EEEEEE}
\definecolor{cgoExtralight}{HTML}{FFFFFF}

%
% Nuevo comando para tablas grandes sin cabecera.
\newcommand{\tablaSinCabeceraConBandas}[5]{%
  \begin{center}
    \tablefirsthead{
      \toprule
    }
    \tablehead{
      \multicolumn{#3}{l}{\small\sl continúa desde la página anterior}\\
      \hline
    }
    \tabletail{
      \hline
      \multicolumn{#3}{r}{\small\sl continúa en la página siguiente}\\
    }
    \tablelasttail{
      \hline
    }
    \bottomcaption{#1}
    \rowcolors[]{1}{cgoExtralight}{cgoLight}

  \begin{xtabular}{#2}
    #5
   \bottomrule
  \end{xtabular}
  \label{tabla:#4}
  \end{center}
}


















\graphicspath{ {./img/} }

% Capítulos
\chapterstyle{bianchi}
\newcommand{\capitulo}[2]{
	\setcounter{chapter}{#1}
	\setcounter{section}{0}
	\chapter*{#2}
	\addcontentsline{toc}{chapter}{#2}
	\markboth{#2}{#2}
}

% Apéndices
\renewcommand{\appendixname}{Apéndice}
\renewcommand*\cftappendixname{\appendixname}

\newcommand{\apendice}[1]{
	%\renewcommand{\thechapter}{A}
	\chapter{#1}
}

\renewcommand*\cftappendixname{\appendixname\ }

% Formato de portada
\makeatletter
\usepackage{xcolor}
\newcommand{\tutor}[1]{\def\@tutor{#1}}
\newcommand{\contutor}[1]{\def\@contutor{#1}}
\newcommand{\course}[1]{\def\@course{#1}}
\definecolor{cpardoBox}{HTML}{E6E6FF}
\def\maketitle{
  \null
  \thispagestyle{empty}
  % Cabecera ----------------
\noindent\includegraphics[width=\textwidth]{cabecera}\vspace{1cm}%
  \vfill
  % Título proyecto y escudo informática ----------------
  \colorbox{cpardoBox}{%
    \begin{minipage}{.8\textwidth}
      \vspace{.5cm}\Large
      \begin{center}
      \textbf{TFG del Grado en Ingeniería Informática}\vspace{.6cm}\\
      \textbf{\LARGE\@title{}}
      \end{center}
      \vspace{.2cm}
    \end{minipage}

  }%
  \hfill\begin{minipage}{.20\textwidth}
    \includegraphics[width=\textwidth]{escudoInfor}
  \end{minipage}
  \vfill
  % Datos de alumno, curso y tutores ------------------
  \begin{center}%
  {%
    \noindent\LARGE
    Presentado por \@author{}\\ 
    en Universidad de Burgos --- \@date{}\\
    Tutor: \@tutor{}\\
    Contutor: \@contutor{}\\
  }%
  \end{center}%
  \null
  \cleardoublepage
  }
\makeatother

\newcommand{\nombre}{Segundo González González} %%% cambio de comando
\newcommand{\dni}{20.219.016-S}
\newcommand{\tutorA}{Carlos López Nozal}
\newcommand{\tutorB}{José Francisco Díez Pastor}
\newcommand{\dptotutorA}{Ingeniería Informática}
\newcommand{\areatutorA}{Lenguajes y Sistemas Informáticos}
\newcommand{\titulo}{Investigación, creación y explotación de modelos para el estudio del COVID-19 en pacientes del Servicio Cántabro de Salud}

% Datos de portada
\title{\titulo}
\author{\nombre}
\tutor{\tutorA}
\contutor{\tutorB}
\date{\today}

\begin{document}

\maketitle


\newpage\null\thispagestyle{empty}\newpage


%%%%%%%%%%%%%%%%%%%%%%%%%%%%%%%%%%%%%%%%%%%%%%%%%%%%%%%%%%%%%%%%%%%%%%%%%%%%%%%%%%%%%%%%
\thispagestyle{empty}


\noindent\includegraphics[width=\textwidth]{cabecera}\vspace{1cm}

\noindent D. \tutorA, profesor del departamento de \dptotutorA, área de \areatutorA.

\noindent Expone:

\noindent Que el alumno D. \nombre, con DNI \dni, ha realizado el Trabajo final de Grado en Ingeniería Informática titulado "\titulo". 

\noindent Y que dicho trabajo ha sido realizado por el alumno bajo la dirección del que suscribe, en virtud de lo cual se autoriza su presentación y defensa.

\begin{center} %\large
En Burgos, {\large \today}
\end{center}

\vfill\vfill\vfill

% Author and supervisor
\begin{minipage}{0.45\textwidth}
\begin{flushleft} %\large
Vº. Bº. del Tutor:\\[2cm]
D. \tutorA
\end{flushleft}
\end{minipage}
\hfill
\begin{minipage}{0.45\textwidth}
\begin{flushleft} %\large
Vº. Bº. del Contutor:\\[2cm]
D. \tutorB
\end{flushleft}
\end{minipage}
\hfill

\vfill

% para casos con solo un tutor comentar lo anterior
% y descomentar lo siguiente
%Vº. Bº. del Tutor:\\[2cm]
%D. nombre tutor


\newpage\null\thispagestyle{empty}\newpage




\frontmatter

% Abstract en castellano
\renewcommand*\abstractname{Resumen}
\begin{abstract}
El COVID-19 es una enfermedad respiratoria causada por el coronavirus SARS-CoV-2. Fue identificado por primera vez en la ciudad de Wuhan, China, en diciembre de 2019 y desde entonces se ha propagado a nivel mundial, convirtiéndose en una pandemia.

El impacto del COVID-19 en la sociedad ha sido significativo, afectando la salud pública, la economía global, la educación y la vida cotidiana de las personas. Muchos países han implementado medidas de confinamiento y restricciones en los desplazamientos para frenar la propagación del virus, lo que ha generado impactos socioeconómicos y cambios en la forma en que las personas trabajan, estudian y se relacionan entre sí.

Los síntomas más comunes del COVID-19 incluyen fiebre, tos seca y dificultad para respirar. Sin embargo, algunas personas pueden ser asintomáticas o presentar síntomas leves, mientras que otras pueden desarrollar complicaciones graves, especialmente aquellos con enfermedades subyacentes y personas de edad avanzada. Por todo esto, adquiere gran importancia tener una prueba que identifique qué pacientes tienen COVID-19 lo antes posible y su grado de severidad para poder empezar a tratarlos inmediatamente.

Para este proyecto se cuenta con datos proporcionados por el Instituto de Investigación Sanitaria de Valdecilla (IDIVAL). El objetivo principal de este trabajo será la investigación y creación de modelos de aprendizaje automático que facilite el triaje en la admisión de pacientes con COVID-19.
\end{abstract}

\renewcommand*\abstractname{Descriptores}
\begin{abstract}
COVID-19, clasificación, aprendizaje automático, diagnóstico, triaje, predicción, soporte a la decisión, python.
\end{abstract}

\clearpage

% Abstract en inglés
\renewcommand*\abstractname{Abstract}
\begin{abstract}
COVID-19 is a respiratory disease caused by the SARS-CoV-2 coronavirus. It was first identified in the city of Wuhan, China, in December 2019 and has since spread globally, becoming a pandemic.

The impact of COVID-19 on society has been significant, affecting public health, the global economy, education, and people's daily lives. Many countries have implemented lockdown measures and travel restrictions to slow the spread of the virus, which has generated socioeconomic impacts and changes in the way people work, study and interact with each other.

The most common symptoms of COVID-19 include fever, dry cough, and shortness of breath. However, some people may be asymptomatic or have mild symptoms, while others may develop serious complications, especially those with underlying diseases and the elderly. For all this, it is of great importance to have a test that identifies which patients have COVID-19 as soon as possible and its degree of severity in order to start treating them immediately.

For this TFG there are data provided by the Valdecilla Health Research Institute (IDIVAL). The objective of this work will be the research and creation of a machine learning model that facilitates triage in the admission of patients with COVID-19.
\end{abstract}

\renewcommand*\abstractname{Keywords}
\begin{abstract}
COVID-19, classification, machine learning, diagnosis, triage, prediction, decision support, python.
\end{abstract}

\clearpage

% Indices
\tableofcontents

\clearpage

\listoffigures

\clearpage

\listoftables
\clearpage

\mainmatter
\capitulo{1}{Introducción}

El síndrome respiratorio agudo severo coronavirus 2 (SARS-coV-2), más conocido como COVID-19, tuvo sus primeros casos en la ciudad de Wuhan, en la provincia de Hubei (China) a finales de 2019, en la que se detectaron los primeros casos de una enfermedad respiratoria desconocida. Desde entonces se ha expandido a lo largo del mundo creando una pandemia con más de 676.609.837 de casos confirmados y con un total de 6.881.955 de muertes asociadas. En España, la incidencia de casos confirmados es de 13.770.429 de pacientes con una mortalidad de 119.479 personas.\footnote{Datos extraídos del cuadro de mando de COVID-19 de la Universidad John HopKins (JHU) a fecha 29/06/2023}

\imagen{mapamundial}{Casos COVID-19 en todo el mundo ~\cite{jhu:2023}. }

En Cantabria, pese a ser una comunidad autónoma con una densidad de población media, el número de casos confirmados asciende a 131.501 en tanto que la enfermedad a producido 815 fallecidos.\footnote{Datos extraídos del portal del Servicio Cántabro de Salud (SCS) con ultima actualización a fecha 30/03/2022} 

La prueba más fiable en la detección de la enfermedad es la Reverse Transcription-Polymerase Chain Reaction (RT-PCR) cuyo resultado tarda en devolverse unas horas. Los resultados de la prueba PCR solo ayudan a detectar si un paciente tiene la enfermedad (positivo) o no (negativo).  Además, de esta información, sería interesante intentar predecir la posible evolución de la enfermedad en el paciente. Es por esto de la importancia de desarrollar modelos de aprendizaje automático que ayuden en el triaje de los pacientes de forma rápida para poder controlar la enfermedad en los centros hospitalarios.

\section{Machine Learning en el triaje del COVID-19}

La aparición de la pandemia desencadenada por el COVID-19 produjo muchos casos de pacientes con síntomas en muy poco tiempo, lo que condujo a una falta de recursos hospitalarios. Esto obligó a buscar alternativas en el aprendizaje automático para producir modelos~\cite{patel:2021}\cite{zhao:2020}\cite{kar:2021}\cite{subudhi:2021}\cite{kurstjens:2021} que pudieran ayudar en el triaje de los pacientes que llegasen al hospital y así ofrecer a los profesionales sanitarios una herramienta con la que poder tomar decisiones.

\imagen{ejemplomodelo}{Diagrama esquemático representando el proceso de desarrollo de un modelo de aprendizaje automático. \cite{subudhi:2021}}

El artículo científico que se toma como referencia \textit{Innate and Adaptive Immune Assessment at Admission to Predict Clinical Outcome in COVID-19 Patients}~\cite{sansegundo:2021} evalúa un amplio panel multiparamétrico de anticuerpos de componentes celulares y humorales de la respuesta inmunitaria innata y adaptativa para buscar biomarcadores que pronosticasen el COVID-19.

El conjunto de datos utilizado está formado por muestras tomadas a 155 pacientes al ingreso del hospital y se categorizaron como leves o graves, en el caso de requerir oxigenoterapia. El modelo predictivo que se utilizó es la regresión logística e incluyó las siguientes características: la edad, la ferritina, el dímero D, la suma absoluta de linfocitos, el  \% monocitos no clásicos, C4 y el  \% de $CD8^{+}CD27^{-}CD28^{-}$.

La edad, el dímero D y la ferritina son características utilizadas habitualmente por los científicos en modelos predictivos, tal y como veremos en el apartado \hyperref[trabajosrel]{trabajos relacionados}.

En el presente trabajo, se abordará el triaje de pacientes a partir de marcadores recogidos en la admisión con aprendizaje automático. Replicaremos el modelo de regresión logística utilizado en el artículo original y buscaremos nuevos modelos predictivos que pudieran mejorar sus datos. Los términos marcador, variable y característica son equivalentes y dependen del contexto en el que se utilicen. En entornos hospitalarios se utiliza marcador y variable y en entornos de investigación en ciencias de datos se utilizan característica y variable.
\capitulo{2}{Objetivos del proyecto} \label{capi2}

El objeto de este proyecto es desarrollar modelos de aprendizaje automático que posibiliten a los médicos realizar un triaje en la admisión de pacientes con sintomatología COVID-19 a partir de variables/marcadores que se puedan obtener de forma rápida sin tener que esperar a una prueba PCR positiva y que, además, pueda predecir la gravedad de la enfermedad para permitir conocer que recursos hospitalarios (ventilación mecánica, ingreso en UCI, etc...) va a necesitar cada paciente.

\section{Objetivos principales}

\begin{itemize}
	\item Replicar el trabajo \textit{Innate and Adaptive Immune Assessment at Admission to Predict Clinical Outcome in COVID-19 Patients}~\cite{sansegundo:2021}.
    \item Buscar trabajos relacionados y clasificar las variables, técnicas y herramientas que se utilizaron.
	\item Estudiar los datos proporcionados por el Instituto de Investigación Sanitaria de Valdecilla (IDIVAL) y considerar como procesarlos para poder obtener un un modelo de predicción que mejore los rendimientos del trabajo \textit{Innate and Adaptive Immune Assessment at Admission to Predict Clinical Outcome in COVID-19 Patients}~\cite{sansegundo:2021}.
    \item Utilizar librerías de aprendizaje automático que ayuden a encontrar el mejor método de clasificación de entre diferentes métodos a partir de las variables/marcadores para el Covid-19 y disponibles en el conjunto de datos.
    \item Minimizar las características a utilizar para su utilización en el modelo a desarrollar.
    \item Investigar qué hiperparámetros pueden mejorar el método de clasificación elegido para el modelo final.
\end{itemize}

\section{Objetivos personales}

\begin{itemize}
    \item Poner en práctica los conocimientos obtenidos de minería de datos.
    \item Desarrollar el proyecto  de manera colaborativa en Github.
    \item Documentar la memoria del proyecto con \LaTeX.
\end{itemize}
\capitulo{3}{Conceptos teóricos}

\section{Minería de Datos}

La minería de datos es un proceso de exploración y análisis de grandes conjuntos de datos con el objetivo de descubrir patrones, relaciones y conocimientos útiles. Se utiliza en diversas industrias y disciplinas para tomar decisiones basadas en datos y obtener información valiosa. La cantidad de datos que se produce cada día crece de forma exponencial en todos los aspectos de la vida. Esto nos ha llevado a utilizar técnicas como el aprendizaje automático para poder explotar esos datos y sacarles un rendimiento.

Para poder extraer conocimiento de esta ingente cantidad de datos existen diversas metodologías como CRISP-DM, Scrum, Kanban... Nosotros hemos utilizado la primera para crear los modelos de aprendizaje automático.

\section{CRISP-DM}

La metodología CRISP-DM (Cross-Industry Standard Process for Data Mining) es un enfoque utilizado para abordar proyectos de minería de datos. Aunque originalmente se desarrolló para aplicaciones comerciales, se puede adaptar para analizar y abordar los desafíos relacionados con el COVID-19.

Se compone de 6 fases:
\begin{itemize}
	\item \textbf{Entendimiento del Negocio} Comprensión clara de los objetivos y requisitos relacionados con el proyecto.
	\item \textbf{Entendimiento de los Datos} Recopilación y exploración de datos para comprender su calidad, estructura y características, identificando posibles problemas o limitaciones que nos podrían afectar a los análisis posteriores.
    \item \textbf{Preparación de los datos} Realización de tareas de limpieza y transformación de los datos, imputando datos que pudieran faltar, eliminando duplicados, normalizando dichos datos y seleccionando características relevantes.
    \item \textbf{Modelado} y análisis de datos para responder a las preguntas planteadas en la etapa de comprensión del negocio, incluyendo la construcción de modelos predictivos. En esta fase puede ser necesario volver a la fase anterior para volver a preparar los datos.
    \item \textbf{Evaluación} de los modelos desarrollados, analizando los resultados y comparándolos con los objetivos iniciales. Se considera la precisión, la eficacia y la calidad de los modelos.
    \item \textbf{Despliegue} del modelo construido que haya dado mejor resultado para que pueda ser utilizado con otros datos.
\end{itemize}

\imagen{crisp-dm}{Metodología de datos CRISP-DM \cite{crisp-img:2021}}

A continuación, se proporciona un resumen de cómo se ha aplicado la metodología CRISP-DM a este proyecto:

\subsection{Entendimiento del negocio}
El triaje sanitario es el proceso de evaluar la gravedad de la condición de un paciente y determinar el orden de atención médica en situaciones en las que los recursos son limitados. La implementación de modelos de aprendizaje automático en los centros hospitalarios permitiría ayudar en la clasificación y priorización de pacientes basándose en datos clínicos y síntomas relacionados con la enfermedad. Además, esto permitiría conocer de antemano la necesidad de recursos hospitalarios, como camas de cuidados intensivos o respiradores, y planificar en consecuencia.

El aprendizaje automático puede ser una herramienta útil en el triaje sanitario, pero no debe reemplazar la experiencia y el juicio clínico de los profesionales de la salud. Los modelos de aprendizaje automático deben utilizarse como una herramienta de apoyo a la toma de decisiones, pero siempre debe haber una supervisión y revisión médica adecuada para tomar decisiones finales.

El desarrollo de modelos de aprendizaje automático puede ser de gran utilidad en el triaje de pacientes y en la gestión de recursos en los centros hospitalarios durante la pandemia de la enfermedad. Estos modelos podrían ayudar a controlar la propagación de la enfermedad y mejorar la atención médica de los pacientes.

\subsection{Entendimiento de los datos}
La cohorte de datos han sido proporcionados por el Instituto de Investigación Sanitaria de Valdecilla (IDIVAL) sobre una población de 305 pacientes, cuyas muestras fueron recogidas entre Abril de 2020 y Marzo de 2021 y que tuvieron un resultado positivo en COVID-19. El valor de clasificación es el \textbf{score} de evolución clínica que tiene un valor de 1 a 5, siendo 1=asintomático; 2=tratamiento con gafas nasales; 3=tratamiento con ventimask; 4=Ingreso en UCI; 5=exitus.

El dataset se compone de un archivo excel con las siguientes características:
\begin{itemize}
	\item 420 filas o determinaciones.
	\item 97 columnas o marcadores.
    \item Sólo hay 158 filas con el valor de clasificación score cumplimentado.
\end{itemize}

Investigando los datos, vemos características que no vamos a utilizar debido a qué son fechas o números identificativos de muestras o pacientes que no nos aportan ningún tipo de información.

\subsection{Preparación de los datos}
Creamos un nuevo dataset copia del original y eliminamos las columnas que comentamos en la sección anterior que no nos aportan datos para clasificación.

Asimismo, observamos que el valor de clasificación \textit{score} tiene muchas filas vacías, lo que no nos permitiría realizar la clasificación, por lo que también las eliminamos.

Una vez tenemos las características con las que vamos a trabajar, observamos que tienen muchos valores perdidos, por lo que utilizaremos la \hyperref[imputadatos]{imputación de datos} para otorgar un valor aproximado para ese dato perdido.

\subsection{Modelado}
Aquí se aplican técnicas de modelado y análisis de datos para responder a las preguntas planteadas en la etapa de entendimiento del negocio. En el caso de este proyecto, se han realizado varios modelos, utilizando \hyperref[autosklearn]{Autosklearn} para realizar una búsqueda del clasificador que mejor funciona con el dataset. Los modelos generados con \hyperref[lda]{lda} han sido los dos métodos de clasificación con los mejores resultados.

\subsection{Evaluación}
En esta fase, se evalúan los modelos desarrollados. Se analizan los resultados y se comparan con el trabajo de partida inicial. Si los modelos no cumplen con los requisitos, se pueden realizar ajustes o mejoras.

\subsection{Despliegue}
Los modelos desarrollados se podrán poner a disposición de los profesionales para poder obtener un triaje de pacientes con síntomas COVID-19 a partir de una serie de muestras que se les recoge en el hospital.

\section{Imputación de datos} \label{imputadatos}
La imputación de datos se refiere al proceso de reemplazar los valores perdidos en un conjunto de datos con valores estimados o inferidos. Es una técnica comúnmente utilizada en el análisis de datos para abordar el problema de los valores perdidos, ya que estos pueden afectar la calidad y la validez de los resultados obtenidos.

Existen varios métodos y técnicas, pero en los modelos de este proyecto he utilizado dos:

\begin{itemize}
	\item \textbf{Imputación simple} Consiste en reemplazar los valores perdidos por un único valor, como la media, la mediana o el valor más frecuente del conjunto de datos. Esta técnica es rápida y sencilla, pero no tiene en cuenta la relación entre las variables y puede introducir sesgos en los resultados.
	\item \textbf{Imputación por vecinos más cercanos} Se basa en encontrar observaciones similares a aquellas con valores que faltan y utilizar los valores de las observaciones vecinas para estimar los valores perdidos. Esta técnica es útil cuando los datos tienen una estructura de vecindad o proximidad.
\end{itemize}
\capitulo{4}{Técnicas y herramientas}

A continuación se enumeran y detallan someramente algunas técnicas y herramientas que se ha utilizado en la realización y desarrollo de este proyecto:

\section{Docker}

Docker\cite{docker:2013} es un software libre y de código abierto que automatiza el despliegue de aplicaciones dentro de contenedores de software, proporcionando una capa adicional de abstracción y automatización de virtualización de aplicaciones en múltiples sistemas operativos. En este TFG, he utilizado Docker instalado en un NAS de QNAP para crear un contenedor con Anaconda Distribution y así poder realizar los diversos modelos en python. \url{https://www.docker.com}

\subsection{Docker Hub}

Es un servicio en línea proporcionado por Docker que sirve como repositorio central para imágenes de contenedores Docker. Es un registro público y gratuito donde los desarrolladores y las organizaciones pueden almacenar, compartir y descargar imágenes de contenedores Docker.

\section{Anaconda}

Anaconda\cite{anaconda:2012} es una distribución de Python ampliamente utilizado por los desarrolladores. \url{https://www.anaconda.com}

\subsection{Jupyter Notebooks}

Dentro de la distribución de python de Anaconda se encuentran los cuadernos Jupyter\cite{jupyter:2014}. También ampliamente utilizados, permiten presentar en un mismo documento texto maquetado con Markdown, código en diferentes lenguajes y los resultados de la ejecución de dicho código, pudiendo representar dichos datos en forma de texto o gráficos, lo que representa una herramienta fundamental para la formación y para la explicación paso a paso de códigos de desarrollo. \url{https://www.jupyter.org}

\section{Python}

Es un lenguaje de programación interpretado y de alto nivel. Es conocido por su sintaxis sencilla y legible, lo que lo hace ideal para principiantes. Python\cite{python:1991} es versátil y se puede utilizar para desarrollar una amplia gama de aplicaciones, desde scripts simples hasta aplicaciones web y científicas de alto rendimiento. Tiene una amplia biblioteca estándar que cubre muchas tareas comunes, lo que facilita el desarrollo de aplicaciones sin necesidad de depender de bibliotecas externas. Python es orientado a objetos, lo que permite una programación modular y estructurada. Es interpretado y multiplataforma, lo que significa que el código fuente se ejecuta directamente y puede ejecutarse en diferentes sistemas operativos.

\section{Bibliotecas de Python}

Python cuenta con una amplia variedad de bibliotecas que proporcionan funcionalidades adicionales y extienden las capacidades del lenguaje. Estas bibliotecas cubren una amplia gama de áreas y permiten a los desarrolladores ahorrar tiempo y esfuerzo al aprovechar el trabajo previo de otros programadores.

\subsection{Pandas}

Ofrece estructuras de datos y herramientas de análisis de datos de alto rendimiento. Permite manipular, limpiar y analizar datos de manera eficiente.

\subsection{Numpy}

Proporciona un soporte completo para arreglos multidimensionales y operaciones matemáticas de alto rendimiento. Es ampliamente utilizado en el procesamiento numérico y científico de datos.

\subsection{Matplotlib}

Es una biblioteca de trazado de gráficos 2D que permite crear visualizaciones de datos de alta calidad. Se utiliza comúnmente para generar gráficos, histogramas, diagramas de dispersión y mucho más.

\subsection{Scikit-learn}

Es una biblioteca de aprendizaje automático de código abierto que proporciona algoritmos para tareas comunes, como clasificación, regresión, agrupación y selección de características.

\subsection{StratifiedKFold}

Es una técnica de validación cruzada utilizada en el aprendizaje automático para evaluar y validar modelos de clasificación. Es una extensión de la validación cruzada k-fold estándar, pero con la adición de la estratificación de las muestras en cada pliegue. La estratificación se refiere a que cada pliegue o partición de los datos de entrenamiento y prueba mantiene la misma proporción de clases o etiquetas que el conjunto de datos original.

\subsection{Pipeline}

Es una secuencia ordenada de pasos que se aplican a los datos para realizar tareas de preprocesamiento, transformación y modelado. Un pipeline facilita la organización y automatización del flujo de trabajo en el desarrollo de modelos de aprendizaje automático.

\subsection{Seaborn}

Es una biblioteca de visualización de datos basada en Matplotlib y es ampliamente utilizada en el análisis exploratorio de datos y la creación de gráficos estadísticos. Proporciona una interfaz más sencilla y atractiva visualmente para crear gráficos informativos y estéticamente agradables.

\subsection{Autosklearn} \label{autosklearn}

Es una biblioteca de aprendizaje automático automatizado (AutoML). Proporciona una interfaz fácil de usar que automatiza el proceso de selección de algoritmos, preprocesamiento de datos, ajuste de hiperparámetros y generación de modelos de aprendizaje automático.

\subsection{Pipeline Profiler}

Es una herramienta que se utiliza en el contexto del aprendizaje automático y los pipelines para analizar y comprender el rendimiento y el comportamiento de un pipeline de manera detallada. Proporciona información valiosa sobre cómo se están ejecutando los diferentes componentes del pipeline y ayuda a identificar cuellos de botella y posibles áreas de mejora en términos de tiempo de ejecución y eficiencia. Al utilizar esta información, se pueden realizar ajustes y optimizaciones para mejorar el rendimiento y la eficiencia del pipeline.

\subsection{Optuna}

Proporciona un enfoque basado en pruebas para encontrar automáticamente los mejores valores de hiperparámetros para un modelo de aprendizaje automático. Optuna ayuda a simplificar y automatizar este proceso al buscar de manera eficiente el espacio de hiperparámetros para encontrar la mejor configuración.

\subsection{Shap}

SHAP (SHapley Additive exPlanations) es utilizada para explicar las predicciones de modelos de aprendizaje automático. Proporciona una forma de entender y descomponer la contribución de cada característica o variable en el resultado de la predicción de un modelo. Proporciona una forma intuitiva y cuantitativa de entender cómo las características individuales influyen en las predicciones del modelo, lo que puede ser útil para la depuración de modelos, la toma de decisiones y la confianza en los resultados obtenidos.

\subsection{Regresión Logística}

Es un algoritmo de aprendizaje automático supervisado utilizado para predecir la probabilidad de que una variable binaria dependiente esté presente en función de variables independientes. Aunque el nombre incluye la palabra "regresión", en realidad es un algoritmo de clasificación.

\subsection{Lda} \label{lda}

Linear Discriminant Analysis (LDA) es un algoritmo de aprendizaje supervisado utilizado para la clasificación y reducción de dimensionalidad, que busca encontrar una proyección lineal óptima para maximizar la separabilidad entre diferentes clases.

\section{Git}

Es un sistema de control de versiones distribuido ampliamente utilizado para el desarrollo de software y el seguimiento de cambios en archivos y proyectos. 

\subsection{Github}

Es una plataforma en línea basada en Git que permite a los desarrolladores almacenar, colaborar, gestionar y compartir proyectos de software de manera eficiente. Es uno de los servicios de alojamiento de repositorios más populares y ampliamente utilizados en la comunidad de desarrollo de software.

\section{\LaTeX}

Es un sistema de composición de documentos ampliamente utilizado para crear documentos de alta calidad, especialmente en ámbitos académicos, científicos y técnicos. A diferencia de los procesadores de texto tradicionales, LaTeX se enfoca en la estructura y el contenido del documento en lugar del formato visual, lo que permite obtener resultados profesionales y consistentes. Se utiliza comúnmente para crear documentos científicos, tesis académicas, artículos de investigación, informes técnicos, libros y presentaciones.

\subsection{Overleaf}

Es una plataforma en línea que permite a los usuarios crear, editar y colaborar en documentos LaTeX de manera colaborativa. Es especialmente popular entre la comunidad académica y científica debido a su facilidad de uso y la capacidad de trabajar en documentos LaTeX sin la necesidad de instalar software adicional.
\capitulo{5}{Aspectos relevantes del desarrollo del proyecto}

Después de la pandemia que sufrimos por el virus COVID-19 y, por ser trabajador en un hospital, habiendo sido testigo de forma directa del caos que se produjo en los hospitales por la acumulación masiva de casos y la dificultad de diagnosticar esta enfermedad de forma rápida, me propuse investigar de qué forma podría contribuir para poder paliar mejorar esta situación.

Realizando una investigación de los artículos científicos publicados acerca de modelos predictivos de COVID-19, encontré el artículo \textit{Innate and Adaptive Immune Assessment at Admission to Predict Clinical Outcome in COVID-19 Patients}\cite{sansegundo:2021} de investigadores del IDIVAL y me puse en contacto con ellos para mostrar interés en experimentar con su dataset aplicándole técnicas de minería de datos para intentar mejorar su resultado.

\section{Estado del arte}

Se realizó una investigación de los diferentes artículos encontrados relacionados con el desarrollo de modelos de aprendizaje automático que pudieran ayudar en el diagnóstico de pacientes con síntomas equivalentes a los desarrollados por el COVID-19.

Se confeccionó un excel con una pestaña con el listado de los artículos escogidos, fecha de publicación, descripción y el motivo del interés al escogerlo. En las demás pestañas se fueron recogiendo de cada artículo las variables escogidas para los experimentos, así como las técnicas y las herramientas utilizadas a la hora de construir los modelos. En la tabla \ref{tabla:vartecher} se expone un balance de las más usadas:

\tablaSinColores{Variables, técnicas y herramientas más utilizadas}{p{3.8cm} p{5.3cm} p{2.6cm}}{3}{vartecher}
{Variables & Técnicas & Herramientas\\}{
Age (11) & AUC ROC (11) & Python (11)\\
Suma de Linfocitos (8) & Regresión Logística (9) & R (3)\\
D-Dimer (7) & Random Forest (6) & \\
Ferritin (4) & XGB (6)  & \\
C Reactive Protein (4) & Extra Trees (3)  & \\
}

En el siguiente capítulo \hyperref[trabajosrel]{6. Trabajos relacionados} se desarrolla más esta información de los artículos en dos tablas.

Tanto el libro excel en el que se recogen el listado de los artículos y el estudio de sus variables, técnicas y herramientas; así como los artículos descargados en formato pdf, se encuentran en el repositorio del trabajo en Github: \href{https://github.com/sgg0008/ml-COVID-19/tree/main/1. Trabajos Relacionados}{1. Trabajos Relacionados}

\section{Preparación del entorno de trabajo}

Para el desarrollo de los cuadernos Jupyter de Anaconda y la ejecución de los mismos, se ha utilizado la dockerización de Anaconda bajo un entorno hardware en un NAS QNAP TS-251+ con un procesador Intel de 4 núcleos y 16Gb de RAM.

\imagen{jupyterdocker}{Jupyter Docker Stacks \cite{jupydock-img:2023}}

Bajo la aplicación \textit{Container Station} se instaló la imagen del contenedor \textit{jupyter/tensorflow-notebook} \cite{jupyterdocker:1991} que se descargó de la plataforma Docker Hub. En la instalación se añadió la creación de un volumen de almacenamiento persistente para poder ir guardando de forma segura los cuadernos que se iban realizando. Este contenedor se escogió porque ya incorpora gran parte del ecosistema de bibliotecas científicas de python; otras bibliotecas fueron instaladas en el arranque del contenedor ya que son utilizadas en este trabajo, como son Autosklearn, Pipeline profiler, Shap y Optuna.

La elección de utilizar como entorno de trabajo los cuadernos jupyter dockerizados es porque proporcionan una serie de ventajas:
\begin{itemize}
    \item Portabilidad: Es fácilmente portable entre diferentes sistemas operativos y entornos de desarrollo, lo que evita problemas de configuración y compatibilidad.
    \item Reproducibilidad: Se crean imágenes de contenedores que contienen todas las dependencias y configuraciones específicas de tu Jupyter Notebook, lo que facilita la reproducción de resultados y la colaboración.
    \item Aislamiento: Los paquetes o bibliotecas instalados en el contenedor no afectarán a tu sistema host, y viceversa.
    \item Escalabilidad: Permite manejar grandes conjuntos de datos y ejecutar cálculos intensivos de manera eficiente utilizando recursos distribuidos.
    \item Administración simplificada: Docker facilita la gestión de múltiples Jupyter Notebooks y entornos. Puedes crear y administrar diferentes contenedores para proyectos específicos o diferentes configuraciones de entorno.
    \item Flexibilidad: Permite personalizar y configurar tu entorno para instalar las bibliotecas y herramientas necesarias, configurar extensiones y temas, y ajustar los recursos del contenedor para optimizar el rendimiento.
\end{itemize}

\section{Cohorte de datos}

La cohorte de datos ha sido proporcionada por el Instituto de Investigación Sanitaria de Valdecilla (IDIVAL) sobre una población de 305 pacientes, cuyas muestras fueron recogidas entre Abril de 2020 y Marzo de 2021. Está constituida por las variables que aparecen en la tabla \ref{tabla:variablesdataset}

\tablaSinColores{Variables del dataset}{p{3.3cm} p{8.9cm}}{2}{variablesdataset}
{Variable & Descripción\\}{IDInm & Identificacion de la muestra\\
NH & Número de historia clínica\\
AntiN & Respuesta anti proteína N de SARS-CoV-2\\
AntiS & Respuesta anti proteína S de SARS-CoV-2\\
AntiM & Respuesta anti proteína M de SARS-CoV-2\\
age & Edad\\
gender & Género\\
score & Score de evolución clínica\\
Line & Timepoint\\
Timepoint & Pacientes en función de covid y tiempo de seguimiento\\
IL6 & Niveles de interleucina 6 en suero\\
ferritina & Niveles de ferritina sérica\\
troponina & Niveles de troponina sérica\\
LDH & Niveles de lactato deshidrogenasa sérico\\
PCR & Niveles de proteína C reactiva sérica\\
procalcit & Niveles de procalcitonina sérica\\
DimD & Niveles de Dímero D en suero\\
fechaAnalisis & Fecha de análisis\\
fechaInicioSintomas & Fecha de Inicio de Síntomas\\
fechaPCRpos & Fecha de PCR positiva\\
fechaIngreso & Fecha de Ingreso\\
fechaIngresoUCI & Fecha de Ingreso UCI\\
fechaAltaUCI & Fecha de alta en UCI\\
fechaAlta & Fecha de Alta general\\
fechaExitus & Fecha de exitus\\
P3 & \% de Linfocitos T CD3\\
P4 & \% de Linfocitos T CD4\\
P8 & \% de Linfocitos T CD8\\
ratio & Ratio CD4/CD8\\
P19 & \% de Linfocitos B CD19\\
P16 & \% de Linfocitos NK (CD16CD56)\\
PNKT & \% de Linfocitos NKT (CD3CD16CD56)\\
LINF ABS & Linfocitos absolutos\\
3 & Linfocitos T CD3 absolutos\\
4 & Linfocitos T CD4 absolutos\\
8 & Linfocitos T CD8 absolutos\\
19 & Linfocitos B CD19 absolutos\\
NK & Linfocitos NK absolutos\\
NKT & Linfocitos NKT absolutos\\
IgG & Niveles séricos IgG\\
IgM & Niveles séricos IgM\\
IgA & Niveles séricos IgA\\
C3 & Niveles séricos C3\\
C4 & Niveles séricos C4\\
Pneut & \% de neutrófilos\\
Plinf & \% de linfocitos\\
Pmonoc & \% de monocitos\\
Peos & \% de eosinófilos\\
Pbas & \% de basófilos\\
AbsNeut & Neutrófilos absolutos\\
AbsLinf & Linfocitos absolutos\\
AbsMonoc & Monocitos absolutos\\
AbsEos & Eosinófilos absolutos\\
AbsBas & Basófilos absolutos\\
TH1 & \% de Linfocitos TH1\\
TH117 & \% de Linfocitos TH1/TH17\\
TH17 & \% de Linfocitos TH17\\
TH2 & \% de Linfocitos TH2\\
TC1 & \% de Linfocitos Tc1\\
TC117 & \% de linfocitos Tc1/Tc17\\
TC17 & \% de linfocitos Tc17\\
TC2 & \% de linfocitos Tc2\\
THMEM & \% de linfocitos Thelper memoria\\
THMEM1 & \% de Linfocitos TH1 memoria\\
THMEM117 & \% de Linfocitos TH1/TH17 memoria\\
THMEM17 & \% de Linfocitos TH17 memoria\\
THMEM2 & \% de Linfocitos TH2 memoria\\
TCMEM & \% de Linfocitos Tc memoria\\
TCMEM1 & \% de Linfocitos Tc1 memoria\\
TCMEM117 & \% de linfocitos Tc1/Tc17 memoria\\
TCMEM17 & \% de linfocitos Tc17 memoria\\
TCMEM2 & \% de linfocitos Tc2 memoria\\
LBCXCR5 & \% de LB CXCR5\\
LBCXCR5PD1 & \% de LB CXCR5 PD1\\
LTCXCR5 & Linfocitos T CXCR5\\
LTCXCR5PD1 & Linfocitos T CXCR5 PD1\\
MoClasicos & \% de Monocitos clásicos\\
MoIntermedios & \% de Monocitos intermedios\\
MoNOclasicos & \% de Monocitos no clásicos\\
PNK56high16lo & \% de linfocitos NK CD56high CD16low\\
PNK5616high & \% de linfocitos NK CD56CD16high (citotóxicos)\\
CTL27p28p & \% de LT citotóxicos $CD27^{+}CD28^{+}$\\
CTL27n28p & \% de LT citotóxicos $CD27^{-}CD28^{+}$\\
CTL27p28n & \% de LT citotóxicos $CD27^{+}CD28^{-}$\\
CTL27n28n & \% de LT citotóxicos $CD27^{-}CD28^{-}$\\
TH27p28p & \% de LT helper $CD27^{+}CD28^{+}$\\
TH27n28p & \% de LT helper $CD27^{-}CD28^{+}$\\
TH27p28n & \% de LT helper $CD27^{+}CD28^{-}$\\
TH27n28n & \% de LT helper $CD27^{-}CD28^{-}$\\
CTLDRn38n & \% de LT citotóxicos $HLADR^{-}CD38^{-}$\\
CTLDRp38n & \% de LT citotóxicos $HLADR^{+}CD38^{-}$\\
CTLDRn38p & \% de LT citotóxicos $HLADR^{-}CD38^{+}$\\
CTLDRp38p & \% de LT citotóxicos $HLADR^{+}CD38^{+}$\\
THDRn38n & \% de LT helper $HLADR^{-}CD38^{-}$\\
THDRp38n & \% de LT helper $HLADR^{+}CD38^{-}$\\
THDRn38p & \% de LT helper $HLADR^{-}CD38^{+}$\\
THDRp38 & \% de LT helper $HLADR^{+}CD38^{+}$\\
}

\subsection{Variable a predecir}

La variable a predecir para desarrollar los modelos es el \textbf{score} de evolución clínica atendiendo a la gravedad de la enfermedad en el paciente y distinguiendo en \textit{leve} si el paciente es leve y \textit{moderado/grave} en el caso que el paciente tuviese que recibir una asistencia hospitalaria con ventilación mecánica.

Atendiendo a esta distinción, se observa que el conjunto de datos está bastante balanceado con 73 pacientes leves y 82 moderados/graves.

\section{Réplica del trabajo original}

Como primer acercamiento, se realizó un cuaderno jupyter en python para replicar el trabajo original \textit{Innate and Adaptive Immune Assessment at Admission to Predict Clinical Outcome in COVID-19 Patients}\cite{sansegundo:2021}: un modelo de regresión logística basada sólo en 7 variables de las recogidas en la cohorte de datos: age, ferritin, d-dimer, absolute lymphocite count, C4, \% de $CD8^{+}CD27^{-}CD28^{-}$ y \%  of non-classical monocytes.

Primero se preparó el dataset para coincidir con los datos que se utilizan en el trabajo original, ya que se toman muestras de pacientes entre Abril y Octubre de 2020 y que hayan resultado positivos en una prueba PCR.

Una vez tenemos los datos preparados, comprobamos en la figura \ref{fig:valoresnulos} que existen valores nulos en alguna de las variables, por lo que necesitaremos realizar una imputación de esos datos nulos; siguiendo las directrices que marca el trabajo, haremos una imputación de la media.

\imagen{valoresnulos}{Valores nulos en el dataset}

Se ha utilizado la validación cruzada con la biblioteca StratifiedKFold varias veces, cada vez con una semilla distinta, de forma que el resultado buscado (AUC ROC) es una media de todos los resultados de cada una de las repeticiones que se realizaron. Se han utilizado 5 repeticiones y 3 folds y 10 repeticiones y 10 folds, respectivamente.

En nuestro caso, el mejor resultado obtenido fue con la imputación de la media y 5 repeticiones y 3 folds, con lo que obtuvimos un AUC ROC de \textbf{76,36\%}.

Entendemos que la desviación obtenida con el resultado del trabajo original (78\%) viene dada por el hecho de haber utilizado la validación cruzada en nuestra réplica.

El cuaderno \textit{Réplica trabajo} se encuentra en el repositorio del trabajo en Github: \href{https://github.com/sgg0008/ml-COVID-19/tree/main/2. Cuadernos TFG}{2. Cuadernos TFG}

\section{Búsqueda del clasificador óptimo con Autosklearn}

En la búsqueda de un clasificador óptimo para la cohorte de datos, se ha utilizado la biblioteca de python \textit{Autosklearn} que es una biblioteca de aprendizaje automático (AutoML) que permite automatizar el proceso de construcción de modelos de aprendizaje automático. En la figura \ref{fig:autosklearn} se puede observar su funcionamiento.

\imagen{autosklearn}{Autosklearn en una imagen \cite{autosklearn-img:2020}}

Para llevar una correlación con la réplica del apartado anterior, se han tenido en cuenta dos factores a la hora de realizar los cuadernos:

\begin{itemize}
    \item Utilizar sólo las características de la réplica o todas las características.
    \item Hacer validación cruzada de 5 o 10 repeticiones con \textit{StratifiedKFold}.
\end{itemize}

A partir de esta información, se crearon cuatro cuadernos jupyter con la misma estructura, pero alternando los factores anteriores.

La preparación de los datos es similar al cuaderno de réplica de datos, tan solo varía en el caso de escoger todas las variables.

Una vez creados los dataframes de entrenamiento (X) y de test (y), se convierten los datos a un formato compatible con autosklearn y se dividen en subconjuntos aleatorios con la función de sklearn \textit{train\_test\_split} con un test\_size de 0.3.

A continuación, se configuraaron los modelos de automl para que realicen una búsqueda del mejor clasificador e hiperparámetros en un tiempo de 1 hora, limitando el tiempo por clasificador en 80 segundos, aplicando una estrategia de resampleado con StratifiedKFold (alternando los splits en 5 y 10 repeticiones) utilizando 4 procesos paralelos y estableciendo como métrica de búsqueda el AUC ROC.

El límite de tiempo ocupado por clasificador de 80 segundos, se estableció atendiendo a las diversas pruebas que se realizaron con autosklearn con anterioridad, observando en los resultados el número de algortimos pérdidos por tiempo.

Estos fueron los resultados obtenidos en los 4 cuadernos:

\begin{enumerate}
    \item Con características de la réplica:
    \begin{itemize}
        \item 10 repeticiones: \ref{fig:repliauto10}
        \begin{description}
            \item[Clasificador:] Passive\_agressive
            \item[AUC ROC:] 82,25\%
        \end{description}
        \imagen{repliauto10}{Resultado experimento réplica con 10 repeticiones}
        \item 5 repeticiones: \ref{fig:repliauto5}
        \begin{description}
            \item[Clasificador:] Extra\_trees
            \item[AUC ROC:] 79,83\%
        \end{description}
        \imagen{repliauto5}{Resultado experimento réplica con 5 repeticiones}
    \end{itemize}
    \item Todas las características:
    \begin{itemize}
        \item 10 repeticiones: \ref{fig:todoauto10}
        \begin{description}
            \item[Clasificador:] Gradient\_boosting
            \item[AUC ROC:] 82,07\%
        \end{description}
        \imagen{todoauto10}{Resultado experimento con todas las caract. y 10 repeticiones}
        \item 5 repeticiones: \ref{fig:todoauto5}
        \begin{description}
            \item[Clasificador:] Lda
            \item[AUC ROC:] 83,78\%
        \end{description}
        \imagen{todoauto5}{Resultado experimento con todas las caract. y 5 repeticiones}
    \end{itemize}
\end{enumerate}

Una vez reajustado el modelo, se utilizó la biblioteca \textit{Pipeline profiler} \ref{fig:pipeprofiler} para comparar los resultados obtenidos en cada experimento realizado. 

\imagen{pipeprofiler}{Pipeline profiler experimento con todas las caract. y 5 repeticiones}

Los cuadernos de \textit{Autosklearn} se encuentran en el repositorio del trabajo en Github: \href{https://github.com/sgg0008/ml-COVID-19/tree/main/2. Cuadernos TFG}{2. Cuadernos TFG}

\section{Optimización de hiperparámetros}

En este apartado, se utiliza la biblioteca Optuna para realizar la optimización de hiperparámetros del modelo LDA (Análisis de Discriminante Lineal). Se define la función objetivo que toma los hiperparámetros sugeridos por Optuna y entrena el modelo en el conjunto de entrenamiento, luego se evalúa su rendimiento en el conjunto de prueba. Optuna realiza la búsqueda de hiperparámetros y guarda los mejores hiperparámetros encontrados.

Además, se utiliza la biblioteca Shap para realizar una exploración de importancia de características del modelo entrenado \ref{fig:shap}. Se crea un objeto Explainer con el mejor modelo y se calculan los valores SHAP para el conjunto de prueba. Finalmente, se visualiza la importancia

\imagen{shap}{Contribución media de las características de cada clase}

Los cuadernos de \textit{Optimización de hiperparámetros} se encuentran en el repositorio del trabajo en Github: \href{https://github.com/sgg0008/ml-COVID-19/tree/main/2. Cuadernos TFG}{2. Cuadernos TFG}
\capitulo{6}{Trabajos relacionados} \label{trabajosrel}

\section{Búsqueda de trabajos relacionados}
Investigación en los principales buscadores de artículos científicos acerca de documentos publicados relacionados con palabras como COVID-19, clasificación, aprendizaje automático, triaje y score. De entre todos, se discriminaron por los artículos que se basaran en \textit{datasets} que contuviesen marcadores recogidos a los pacientes para el diagnóstico del COVID-19.

En la tabla \ref{tabla:trabajosrelacionados}, se enumeran el listado de los trabajos relacionados con la fecha de publicación y una breve descripción.

\tablaConColores{Trabajos relacionados con este proyecto}{p{4cm} p{1.7cm} p{6cm}}{3}{trabajosrelacionados}
{Título & Fecha & Descripción\\}{ 
Innate and Adaptive Immune Assessment at Admission to Predict Clinical Outcome in COVID-19 Patients \cite{sansegundo:2021} & 29/07/2021 & Predicción de la severidad de la enfermedad en la admisión del paciente categorizada por los requisitos de oxigenoterapia\\
Machine-learning-based COVID-19 mortality prediction model and identification of patients at low and high risk of dying \cite{banoei:2021} & 08/09/2021 & Modelo de predicción de mortalidad para pacientes COVID-19 hospitalizados, así como una clasificación de pacientes para verificar los grupos de bajo y alto riesgo\\
Deep forest model for diagnosing COVID-19 from routine blood tests \cite{aljame:2021} & 17/08/2021 & Uso de técnicas de machine learning basadas en datos clínicos y/o de laboratorio para la detección temprana de COVID-19\\
Development of a severity of disease score and classification model by machine learning for hospitalized COVID-19 patients \cite{marcos:2021} & 21/04/2021 & Modelo de machine learning para identificar de manera temprana a los pacientes que morirán o requerirán ventilación mecánica durante la hospitalización a partir de las características clínicas y de laboratorio obtenidas en el momento del ingreso\\
Machine learning based predictors for COVID-19 disease severity \cite{patel:2021} & 25/02/2021 & Clasificación mediante random forest de la severidad en los pacientes con COVID-19 para calcular si necesitarán ventilación mecánica o/y ingreso en UCI\\
Prediction model and risk scores of ICU admission and mortality in COVID-19 \cite{zhao:2020} & 30/07/2020 & Desarrolla scores basados en las características clínicas en el momento de la admisión para predecir el ingreso y la mortalidad en UCI en pacientes con COVID-19\\
Scoring systems for predicting mortality for severe patients with COVID-19 \cite{shang:2020} & 15/07/2020 & Definición de sistemas de score para predecir el ingreso en UCI y/o la mortalidad en pacientes de COVID-19\\
A Multimodality Machine Learning Approach to Differentiate Severe and Nonsevere COVID-19: Model Development and Validation \cite{chen:2021} & 07/04/2021 & Uso de machine learning para diferenciar con precisión los tipos clínicos de COVID-19 graves y no graves en función de múltiples características médicas y proporcionar predicciones fiables del tipo clínico de la enfermedad\\
Multivariable mortality risk prediction using machine learning for COVID-19 patients at admission (AICOVID) \cite{kar:2021} & 17/06/2021 & Desarrollar y validar puntuaciones de riesgo de mortalidad individualizadas basadas en los datos clínicos y de laboratorio anonimizados en el momento del ingreso y determinar la probabilidad de Muertes a los 7 y 28 días\\
A systematic review of prediction models to diagnose COVID-19 in adults admitted to healthcare centers \cite{locquet:2021} & 18/06/2021 & Identificar, comparar y evaluar el desempeño de los modelos de predicción para el diagnóstico de COVID-19 en pacientes adultos en un entorno de atención médica\\
Machine learning is the key to diagnose COVID-19: a proof-of-concept study \cite{gangloff:2021} & 30/03/2021 & Desarrollar y evaluar modelos de aprendizaje automático utilizando datos clínicos y de laboratorio de rutina para mejorar el rendimiento de RT-PCR y CT de tórax para el diagnóstico de COVID-19 entre pacientes hospitalizados después de una urgencia\\
Comparing machine learning algorithms for predicting ICU admission and mortality in COVID-19 \cite{subudhi:2021} & 21/05/2021 & Compara el rendimiento de 18 algoritmos de aprendizaje automático para predecir la admisión y la mortalidad en la UCI entre pacientes con COVID-19\\
Rapid identification of SARS-CoV-2-infected patients at the emergency department using routine testing \cite{kurstjens:2021} & 29/06/2020 & Desarrollar un algoritmo para evaluar rápidamente el riesgo de una persona de infección por SARS-CoV-2 en el servicio de urgencias\\
Machine Learning for Mortality Analysis in Patients with COVID-19 \cite{sanchez:2021} & 12/11/2020 & Modelos de ML para encontrar reglas de decisión interpretables para estimar el riesgo de mortalidad de los pacientes que se pueden obtener del árbol de decisiones y que puede ser crucial en la priorización de la atención y los recursos médicos\\
}

\section{Clasificación de variables, técnicas y herramientas}

Una vez escogidos los trabajos sobre los que se basa la investigación para este proyecto, aparece la necesidad de clasificar de cada trabajo qué variables, técnicas y herramientas utilizaron los diversos autores en su investigación. De esta manera, se da la posibilidad de ver qué variables son las más usadas en las investigaciones, así como qué técnicas más se repiten y cuáles son las herramientas que se utilizan.

En la siguiente tabla \ref{tabla:clasificavariables} se expone qué variables con qué técnicas y herramientas se utilizaron en los trabajos relacionados.

\tablaConColores{Clasificación de variables, técnicas y herramientas}{p{5cm} p{3.4cm} p{3.4cm}}{3}{clasificavariables}
{Variables & Técnicas & Herramientas\\}{ 
Age, Ferritin, D-dimer, C4, \% de $CD8^{+}CD27^{-}CD28^{-}$, Suma absoluta de Linfocitos, \% de Monocitos no clásicos & Regresión Logística & GraphPad (Prism)\\
AST, WBC, Lymphocytes, Neutrophils, GGT, AGE, Basophils, Eosinophils, ALT, Platelets, gender, CRP, ALP, LDH, Monocytes & Extra trees, XGBoost, LightGBM, SHapley Additive exPlanations (SHAP) & Python (sklearn) \\
Old age, coronary heart disease (CHD), percentage of lymphocytes (LYM\%), procalcitonin (PCT), D-dimer (DD) & LASSO regression, Roc curve & Python\\
demographic variables (including age and sex), individual comorbidities and Charlson Comorbidity Index, chronic medical treatment, clinical characteristics, physical examination parameters, biochemical parameters & Random forest, Xgboost, Logistic regression, AUC & Python (sklearn, xgboost, eli5) \\
Leukocytes, Neutrophils, Lymphocytes, Eosinophil, Hemoglobin, Hematocrit, Platelets, ESR, BUN, Creatinine, Na, K, Ferritin, CRP, PCT, Lactate, Troponin, CK, BNP, LDH, Fibrinogen, ALT, AST, Albumin, D-dimer, Bilirubin, Prothrombin, APTT, pH, PaCo2, FiO2\_lab, Bicarbonate, CAD, Diabetes, Age > 65, AMS, Dementia, Nursing home, Q2 saturation < 88, yno2, Consolidation, Hypertension, Atrial fibrillation, Alcohol, Chest pain, Peripheral vascular disease, Stroke, Headache, Dyspnea, CRP, Lactate, Smoking & SIMPLS, AUC (Area under ROC curve), Decission Tree &	Python \\
age, sex, travel, contact history, and co-morbidities, fever, dyspnea, respiratory rate, and blood oxygen saturation (SpO2), RT-PCR, InterLeukin-6, D-Dimer, complete blood count, lipase, and C-reactive protein (CRP) & Random Forest, multilayer perceptron, support vector machines, gradient boosting, extra tree classifier, adaboost, Regresión, Validación cruzada quíntuple, AUC & Python \\
lactate dehydrogenase, procalcitonin, pulse oxygen saturation, smoking history, lymphocyte count, heart failure, chronic obstructive pulmonary disease, heart rate, age & Regresión Logística, AUC & TRIPOD \\
age, hypertension, cardiovascular disease, gender, diabetes, dimerized plasmin fragment D, high sensitivity troponin I, absolute neutrophil count, interleukin 6, lactate dehydrogenase & Random Forest, Predictive mean matching (PMM), Regresión Logística, AUC & R, Python \\
Age, Male Gender, Respiratory Distress, Diabetes Mellitus, Chronic Kidney Disease, Coronary Artery Disease, respiratory rate > 24/min, oxygen saturation below 90\%, Lymphocyte\% in DLC, INR, LDH, Ferritin & eXtreme Gradient Boosting (XGB), Random Forest, Regresión Logística, ‘Time to event’ using Cox Proportional Hazard Model, AUC ROC, Kaplan Meier (KM) Plots & Python \\
 & Prediction model study Risk of Bias Assessment Tool (PROBAST), XGBoost, Regresión Logísitica, AUROC & Python \\
Cough, Hyperthermia, Myalgias, Asthenia, Diarrhea, Confusion, Furosemid (usual treatment), Base excess, Lactates, Red blood cell count, Mean platelet volume, Leukocytes, Neutrophils, Platelet count, Eosinophils percentage, Basophils percentage, Lymphocytes, Monocytes, Ionogram, Potassium, Phosphor, Alanine aminotransferase, International normalized ratio, D-Dimer & binary logistic regression, random forest, artificial neural network, AUC, K-fold cross-validation & R-Studio (Dplyr, Purrr, missForest, Caret, corrplot, randomForest, neuralnet, pROC) \\
C-reactive protein, neutrophil percentages, lactate dehydrogenase, first respiratory, lower xygen saturation, lymphocyte percentages, estimated glomerular filtration rate (eGFR) < 60, high neutrophil percentage, high serum potassium, low lymphocyte percentages, high procalcitonin, D-dimer & AdaBoostClassifier, BaggingClassifier, GradientBoostingClassifier, RandomForestClassifier, XGBClassifier, ExtraTreesClassifier, LogisticRegression, DecisionTreeClassifier, LinearDiscriminantAnalysis, QuadraticDiscriminantAnalysis, MLPClassifier, PassiveAggressiveClassifier, Perceptron, LinearSVC, KNeighborsClassifier, GaussianNB & Python \\
Age, Gender, C-reactive protein, lactate dehydrogenase, ferritin, absolute neutrophil, lymphocyte counts & AUROC & Python, Excel 2010 \\
Patient ID, Age and gender, COVID diagnostic (confirmed/pending confirmation), ER date in, ER specialty, ER diagnostic, and destination after ER, First and last constant measurements in the ER (heart rate, temperature, minimum and maximum arterial pressure, O2 saturation in blood), Admission date to the hospital, ICU date in, ICU date out, and number of days in the ICU (if applicable), Discharge date and destination (home/deceased/transferred to other hospital/voluntary discharge/transferred to a socio-sanitary center) & Logistic regression, Survival Analisys, Decision tree, Random Forest, Bayessian networks, Biclustering, AUC & R (Caret, e1701, ggplot2), Python (sklearn, pandas, numpy  matplotlib) \\
}
\capitulo{7}{Conclusiones y Líneas de trabajo futuras}

Se exponen a continuación las conclusiones que se extraen de la realización de este proyecto, así como las posibles líneas de trabajo que se pueden considerar en un futuro.

\section{Conclusiones}

\subsection{Relacionadas con los resultados}

Los resultados obtenidos en los diferentes modelos realizados son parecidos a los obtenidos por el trabajo que se replicó. Esto se debe en parte a que se realizó sobre una pequeña cantidad de muestras y, por otra parte, a qué dichos datos no estaban del todo completos.

Estos resultados de los modelos se pueden mejorar con una mayor cantidad de muestras recogidas y con una mayor calidad en la recogida de los datos.

\subsection{Técnicas}
En este proyecto se ha realizado un seguimiento experimental riguroso:
\begin{itemize}
    \item Se realizó un estudio pormenorizado de los trabajos relacionados y la utilización que en ellos se hace de características, técnicas y herramientas, que deparó un mejor conocimiento de la realización de experimentos de aprendizaje automático con marcadores clínicos.
    \item Se prepararon los datos de la cohorte proporcionada, ya que contenían variables que no aportaban una información clara para nuestra investigación (como las fechas) y algunos valores nulos que fueron subsanados con la imputación de datos.
    \item La realización de la réplica del trabajo \textit{Innate and Adaptive Immune Assessment at Admission to Predict Clinical Outcome in COVID-19 Patients}\cite{sansegundo:2021} permitió conprender el porqué del trabajo original y el modelo de regresión logística aplicado.
    \item La utilización de autosklearn conllevó un estudio de esta biblioteca de python para poder entender qué valores introducir en el modelo automl que produjeran mejores resultados.
    \item Con la optimización de hiperparámetros con la biblioteca Optuna se pudo comprender cómo mejorar los experimentos en base a la realización de muchos otros experimentos variando los hiperparámetros.
    \item Conocimiento de biblioteca Shap para explorar la importancia de características del modelo de aprendizaje automático
\end{itemize}

\section{Líneas de trabajo futuras}

Como cualquier otro proyecto de índole experimental, una de las líneas de trabajo futuro es continuar con la experimentación. La tecnología avanza de forma imparable cada día y es por esto que, las bibliotecas científicas de software se actualizan y mejoran, las técnicas de aprendizaje automático se depuran y los sistemas dónde ejecutar las herramientas para los experimento no dejan de crecer en rendimiento. Todos estos factores hacen que un experimento realizado, pueda mejorarse de forma sustancial en el tiempo.

Otra línea clara a seguir es la distribución de los modelos para la utilización por parte de cualquier usuario, por ejemplo en un entorno web como se hizo en el artículo \textit{Development of a severity of disease score and classification model by machine learning for hospitalized COVID-19 patients} \cite{marcos:2021} con la distribución de la página \href{https://covid19salamanca-score.herokuapp.com/}{Herokuapp}; en esta aplicación, un profesional sanitario puede obtener una ayuda para identificar de manera temprana a los pacientes que morirán o requerirán ventilación mecánica durante la hospitalización a partir de las características clínicas y de laboratorio que debe introducir.


\bibliographystyle{plain}
\bibliography{bibliografia}


\end{document}
